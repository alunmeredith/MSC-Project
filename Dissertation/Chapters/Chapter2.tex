% Chapter Template

\chapter{Literature Review} % Main chapter title

\label{Chapter2} % Change X to a consecutive number; for referencing this chapter elsewhere, use \ref{ChapterX}

%----------------------------------------------------------------------------------------
%	Literature Review
%----------------------------------------------------------------------------------------
\section{History}

The study of information networks has seen a growing interest in the past 20 years as the networked systems become more and more popular and integral to modern society through revolutions in communication with email and mobile phones, revolutions in content consumption with the internet and revolutions in system architecture with the internet of things and agent based systems. 

Until recently many of these networks were treated as variations of random graphs, where edges (links) are attached to a set of vertices (nodes) sequentially through a random process \cite{erdds1959random,erd6s1960evolution}. However many networks have been shown to display characteristics not present in random graphs so different models with different attachment mechanisms are required. 

A 'Complex Network' is a macroscopic description of systems in which emergent macroscopic behaviour is produced from the small scale interactions, for example the climate is emergent from the interaction of different weather systems \cite{steinhaeuser2011complex}. These networks are loosely divided into two main classes: 'small world' and 'scale-free' networks. Small world networks are known for their short maximum path length between two nodes, producing characteristics like the 'Six Degrees of Separation'. These networks are highly clustered with long distance links between those clusters \cite{watts1998collective}. Scale-free networks however typically are not strongly clustered and are known for the distribution of edges across nodes (degree) following a power-law $f(x) \sim x^{-\alpha}$, this is sometimes also known by Zipf's law or a Pareto distribution \cite{zipf1935psycho}. This distribution has the property that there is no characteristic scale for the network. A typical scale means that each node has about the same consistent numbers of links producing structures similar to lattices so having no typical scale produces very unstructured networks. 

Scale-free networks are often characterised by a 'rich get richer' phenomenon, known as the Matthew effect \cite{merton1968matthew}. Both Price and Barabási present a 'preferential attachment' mechanism to model this effect where new edges are attached to nodes with probability as a linear function to the number of edges of that node to produce a power-law distribution \cite{albert2002statistical,price1976general}. 

\section{Bibliographic Networks}

Bibliographic networks are networks of authors and/or their works. Examples of these include collaboration networks, where links are formed from instances of author collaboration and citation networks where authors cite other works to be related. Academic citation and collaboration networks were some of the earliest identified as complex networks \cite{albert2002statistical}. 

The statistical nature of many complex networks, including patent networks is still being debated. This forms two fundamental questions: what is the functional form of the underlying distribution and what mechanisms form this distribution, these primarily try to identify power-law distributions and preferential attachment mechanisms although more recently research has extended preferential attachment with additional mechanisms such as ageing terms and fitness models. Answering the first question is difficult as log-normal curves can be very similar to power-law curves under the correct parameters. Clauset et al. describes a method for differentiating between these distributions using bootstrapping methods to discount unlikely distributions and log-likelihood ratios to compare two functional forms \cite{clauset2009power}. Although preferential attachment produces power-laws, a power-law is not sufficient to show presence of preferential attachment. Power-laws have been shown to be produced by other mechanisms such as a multiplication of many random processes \cite{newman2005power}. Log-normal distributions have also been shown to be produced by random multiplicative processes but also sub-linear preferential attachment \cite{redner2000random}. Therefore preferential attachment must be identified independently of the distributions functional form.  

In academic citation networks there has been recent research in the mechanics behind citations such as the propagation of errors which suggests 70 - 80\% of citations are copy and pasted from a secondary source \cite{simkin2005stochastic} or the study of redundant edges to show that the majority of references (~70\%) are secondary \cite{clough2015transitive}. Scientific citation networks have been shown to have characteristics consistent with a preferential attachment mechanism and Weibull ageing term \cite{borner2004simultaneous}.

Multi-dimensional networks are a generalisation of networks combining networks of nodes with multiple classes of link. The study of these networks has become one of the fastest growing areas of research in network science offering generalised diagnostics which can account for differences in nature between the links as well new insights \cite{hardtoSpell}.

\section{Patent Networks}

The main differentiating factor between academic citations and those found in patent networks is the stricter legal scope of citations in patents. Patents have a legal requirement to cite all 'prior art' from which the patent is based. This is a more narrow definition of a citation but also more exhaustive because if a 'prior art' is not cited in the application ideally a patent examiner will add it to the document. There is a larger body of research on academic citation networks due to the inherent domain knowledge researchers have with the field and because it was among the first to be identified as scale-free \cite{albert2002statistical}.

In 2007 Cs'ardi et al. published the first paper examining the US patent office database from a graph theoretical perspective \cite{csardi2007modeling}. They did this by by applying a basic model that assumes the attractiveness of a node (rate at which new nodes attach to it) is a function of the age and number of citations of the node. Normalising for the growth of the patent network over time they found the total attractiveness of the system over time could only be replicated with a super-linear preferential attachment model. They also explore the idea that an increase in the number of citations per patent over time has been coupled with a fundamental change in the structure of the network. Finding that the level of stratification starts to increase in alignment with the higher citation rates. 

Valverde et al. \cite{valverde2007topology} also analyses the data-set from a graph-theoretical perspective suggesting a form for the extended power law for the in-degree distribution, this functional form converges to a power law for high orders and an exponential for low orders. They also explore the clustering and modularity of the system. The clustering coefficient being approximately inversely proportional to the number of citations, suggesting a hierarchical structure. Finally they disagree with Cs'ardi et al. finding a linear preferential attachment. 

A recent study builds exposes the danger of relying on simple models studying the patent network as a whole \cite{gress2010properties}. Bernard Gress (2009) investigated the ratios between citations given and received in different technology groups. High number and diversity of citations given was treated as an indication of generality and number of citations received of productivity and originality. He then compared these measures and how they varied over time for different technology categories. He primarily concludes that these categories are fundamentally different therefore research needs to take this into account. 

Byungun Yoon \cite{yoon2004text} built a patent network from weighted term similarities of patent documents rather than bibliometric citations; using standard bag of words methods and comparing these measures to analogies in citation networks. Through inspection they argue that the centrality of their network yields a more relevant approximation of impact because it is less biased by age and preferential attachment mechanisms. 

As patents are a representation of innovation many studies try to measure the impact of a patent. Network based techniques often limit themselves to using the number of citations received as a measure of impact, however there is a lot of debate as to what extent this measure is valid and how these can be improved. 

\subsection{Innovation Networks}

There is a body of research exploring the analogy between the evolution of innovation and biological evolution. The premise is that each invention is built from the recombination of previous inventions. The two models have their differences, for example there is a limited concept of 'death' in innovation as very old patents may still be cited by new ones and it is hard to think of bibliometric patent networks as direct lineages. 

Yeoun et al. \cite{youn2015invention} explores this idea of invention as a recombination process by looking at the use of technology codes in patents as a proxy for novelty. Technology codes map the technological niche of a patent into categories and subcategories. Patents can have combinations of technology codes. They show that as the number of patents increases the number of new codes being generated falls off while new code combinations maintains a power-law, concluding new technologies has a minimal role relative to recombination. They also show that 40\% of patents use existing combinations vs. new ones suggesting these are incremental improvements. 

Technology code combination distributions do not age in the same way as bibliometric citations, codes appear not to age with 99\% of codes being used at least once every 7 years. 

They also look at the dissimilarity of the codes as a proxy for novelty. If a patent is used in a very different field from its parents it is argued that it is more likely to be a bigger leap in novelty. Categorising patents as either narrow or broad and using count as a metric, we only get a sense how novelty has changed with time and not any of the network factors which may be present here, such as the distribution of novelty could be a power law or the degree of novelty can be a measure of linkage between clusters. 

The limitations of the evolutionary analogy are loosely addressed warning that citations in patents aren't directly related to lineage but about carving a legal niche and there being no good metric of fitness for patents. 

Buchanan et al.\cite{buchanan2011measuring} glosses over some of these limitations, using the number of citations a patent has as an "impact" metric, a proxy for fitness. Prior art citations also function as a proxy for combinatorial lineage. They tell the story of the most cited patents in the network over the past 30 years and show that such a distribution of citations cannot come from random natural selection and therefore must be due to adaptive selection in an evolutionary model of innovation. This argument is an evolutionary perspective on the random network vs. preferential attachment network differentiation. 

They focus on showing this idea more robustly incorporating a multitude of normalisation techniques and simulating a null hypothesis random network model by sampling existing data, rather than building a clean model from scratch. Observing the familiar hallmarks of a fat-tailed distribution they conclude that these "superstars" high impact is due to adaptive features, however they do not address the role of preferential attachment here, how many of the citations received are due to 'rich getting richer' mechanics or due to the intrinsic quality of those innovations. 

Their paper also investigates the dissimilarity of technology codes as a proxy for novelty making the claim that large leaps in novelty are responsible for the largest "impact" patents. However because its scope is only looking at the 20 most cited patents falls short of being able to make such a general argument about the network as a whole. 

Finally Arthur et al.\cite{arthur2014evolution} makes the most direct contrast between the search process of a genetic algorithm and the evolution of technology. In their paper they simulate an evolving population of logical circuits starting from simple logic gates in order to meet a selection of logical needs. Analysing the evolution of the population and resulting network. 

They find many of the typical evolutionary features present such as building blocks being formed as intermediary steps to complex solutions i.e. the building block hypothesis. Sub-optimal solutions slowly become extinct after better solutions emerge.

Complex features are also observed such as a loose power law distribution of edges in the network and avalanches of redundancy as new technologies replace old ones and their dependencies, the size of these redundancies follows a power law showing self-organised criticality. 

The paper incorporates a standard genetic algorithm; ignoring many of the observed differences between natural selection and the evolution of innovation, such as holding a finite population therefore incorporating the "death" of patents as they are replaced, and use of random selection. Both of these were shown earlier to not be accurate in the patent network, despite this they achieve results similar to observed patent networks, further research could conclude that many of the differences observed naturally arrive from a simple model for example patent ageing could be a result from the saturation of combinatorial space around older technologies.  

\section{Conclusion}

In conclusion there is ongoing research into the statistical and structural nature of the patent network. Although there are some methods designed to analytically differentiate between functional forms it appears that investigations using these methods are absent from the patent network and there are contradictory conclusions into the nature of preferential attachment in the network. 

Finally there are significant structural changes between technology groups and evidence of structural changes within the network over time. 

% Technology codes

%There are three main directions in which Patent Networks are studied, as a branch of already existing Price model networks, through natural language processing techniques and through an analogy to evolutionary processes. 

%We have seen how efforts have been made to link the evolution of innovations to a Darwinian evolutionary models, that despite some success a lot more effort needs to be done to incorporate the differences between natural and technological evolution into models. linking these more strongly with networks models can be a key to understanding the mechanics of the innovation of evolution.